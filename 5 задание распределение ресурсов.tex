\documentclass[russian,utf8,nocolumnxxxi,nocolumnxxxii]{eskdtext}
\usepackage[T1,T2A]{fontenc} 
\usepackage[utf8]{inputenc}
\usepackage[english,ukrainian,russian]{babel}
\usepackage{amssymb,amsmath}
\usepackage[shortlabels]{enumitem}
\usepackage{tikz}
\usepackage{pgfplots}
\usepackage{siunitx}
\usepackage[american,cuteinductors,smartlabels]{circuitikz}
\usepackage[backend=biber]{biblatex}
\addbibresource{error_estimation_otchet.bib}
\usepackage[]{hyperref}
\hypersetup{colorlinks=true}
\usepackage{textcomp}
\newcommand{\No}{\textnumero}
\ESKDdepartment{Федеральное агентство по образованию}
\ESKDcompany{Санкт-Петербургский государственный электротехнический университет "ЛЭТИ"}
\ESKDtitle{Пояснительная записка к Курсовой работе}
\ESKDsignature{Вариант №11}
\ESKDauthor{Комиссаров В.Р.}
\ESKDchecker{Прокшин А.Н.}
\ESKDdocName{по дисциплине "Информатика"}
\begin{document}
\maketitle
\newpage
    \section{Задача оптимального распределения неоднородных ресурсов}
    \begin{table}[h]
        \centering
     \caption{Условия поставленой задачи}
\begin{tabular}[c]{|*{6}{c|}}
\hline
Исп. рес-ы & Изд_1 & Изд_2 & Изд_3 & Изд_4 & Наличие\\
\hline
Труд. & 4 & 4 & 1 & 9 & 18\\
\hline
Матер. & 3 & 4 & 5 & 3 & 11\\
\hline
Фин. & 6 & 5 & 8 & 4 & 33\\
\hline
Прибыль & 50 & 40 & 20 & 30\\
\hline
\end{tabular}
\end{table}

Для нахождения оптимального решения воспользуемся функцией \textit{linpro} пакета \textit{SciLab}, где \textit{"p"} - коэф. при неизвестных целевой $f$, \textit{"C"} - матрица неизвествых системы ограничений, а \textit{"b"} содержит свободные члены (\textit {"ci"} и \textit{"cs"} - соответственно нижняя и верхняя границы переменных).

 \vspace{20pt}
 
Система ограничений выглядит следующим образом.
\begin{equation*}
\begin{cases}
  4x_1+4x_2+1x_3+9x_4 \leqslant 18
   \\
  3x_1+4x_2+5x_3+3x_4 \leqslant 11
   \\
  6x_1+5x_2+8x_3+4x_4 \leqslant 33
  \end{cases}
\end{equation*}

\vspace{20pt}

Составляем выражения:

\vspace{20pt}
  
   \textit{f}_{max} = $50x_1+40x_2+20x_3+30x_4$
   
   \vspace{20pt}
   
   \textit{C}= \begin{pmatrix}
   4 \ 4 \ 1 \ 9 \\
   3 \ 4 \ 5 \ 3 \\
   6 \ 5 \ 8 \ 4 
   \end{pmatrix}  \textit{b}= \begin{pmatrix}
   18 \\
   11 \\
   33
   \end{pmatrix} \textit{p}= \begin{pmatrix}
   50 \\
   40 \\
   20 \\
   30
   \end{pmatrix}
   
   \newpage
   
  В итоге были рассчитаны значения, при которых прибыль максимальна:
  
  \textit{f_{max}} = 80 
  
  \textit{larg} = (0;0;55;25;0;10;0)
  
  \textit{x} = (0;1.6;0;0)
  
  Ответ: Прибыль максимальна при производстве 1.6 единиц изделия №2.


\end{document}
